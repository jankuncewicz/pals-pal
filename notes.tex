\documentclass[a4paper]{article}
\usepackage[utf8]{inputenc}
\usepackage[T1]{fontenc}
\usepackage{amsmath}
\usepackage{amssymb}
\usepackage{txfonts}
\usepackage{hyperref}

\begin{document}
	\section{Calculating $\tau$ from ETE}
	\subsection{Calculating zeroes of Bessel functions of the first kind}
	\subsubsection{Initial thoughts}
	Values of Bessel functions of the first kind are
	kindly provided by the Go's mathematics package.
	Because Bessel functions' derivatives can be defined by 
	\begin{equation*}
		\frac{d}{dx}J_n(x) = J_{n-1}(x) - \frac{n}{x}J_n(x)
	\end{equation*}
	or by
	\begin{equation*}
		\frac{d}{dx}J_n(x) = \frac{1}{2}(J_{n-1}(x) - J_{n+1}(x))
	\end{equation*}
	(which might be better because we don't peroform needless division by floating-point numbers),
	we can use approximation techniques for finding zeroes that use derivatives.
	I'm thinking mainly about Newton's or Halley's methodes.

	Recursive sequence in Newton's method is defined in our case by
	\begin{equation*}
		x_{n+1} = x_n - \frac{J_v(x_n)}{J'_v(x_n)} = 
			x_n - \frac{J_v(x_n)}{J_{v-1}(x) - \frac{v}{x_n}J_v(x)} =
			x_n - \frac{2J_v(x)}{J_{v-1}(x) - J_{v+1}(x)}
	\end{equation*}
	The last equation might be actually worse than I innitialy thought
	because we would need to calculate Bessel function 3 times compared to 2 times
	in the second to last equation.

	Halley's method can be defined by
	\begin{equation*}
		x_{n+1} = x_n - \frac{2J_v(x_n)J'_v(x_n)}{2(J'_v(x_n))^2 - J_v(x_n)J''_v(x_n)}
	\end{equation*}
	which looks not as easy by I will check how fast it converges compared to
	Newton's method. Optimal function that calculates derivatives migh be easilly constructed.
	\subsubsection{Code and comparisons}
	\section{A study of the extended Tao-Eldrup model's parameters}
		We are given following equations from \cite{Zaleski2015}:
		\begin{equation}
			P_{nm} = 
				\frac
					{\int\limits_{Z_{nm}\frac{R}{R+\Delta}}^{Z_{nm}}J_m(r)^2rdr}
					{\int\limits_0^{Z_{nm}}J_m(r)^2rdr}
		\end{equation}
		\begin{equation}
			\lambda_{nm} = \lambda_pP_{nm} + \lambda_i(1-P_{nm})
		\end{equation}
		\begin{equation}
			\tau =
				\frac
					{\sum\limits_n\sum\limits_mg_m\mathrm{exp}\left(-\frac{E_{nm}}{kT}\right)}
					{\sum\limits_n\sum\limits_mg_m\lambda_{nm}\mathrm{exp}\left(-\frac{E_{nm}}{kT}\right)}
		\end{equation}
		\begin{equation*}
			E_{nm} = \frac{\hbar^2}{4m_e}\frac{{Z_{nm}}^2}{(R+\Delta)^2}
		\end{equation*}
		where $Z_{nm}$ is the $n$-th node of the Bessel function of the
		first kind $J_m(r)$ and $g_m$ is the statistical weight of the $m$-th state
		($g_0 = 1$, $g_{m>0}=2$).

	\bibliographystyle{abbrvurl}
	\bibliography{refs.bib}
\end{document}