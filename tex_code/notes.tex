\documentclass[a4paper]{article}
\usepackage[utf8]{inputenc}
\usepackage[T1]{fontenc}
\usepackage{amsmath}
\usepackage{amssymb}
\usepackage{txfonts}
\usepackage{hyperref}
\usepackage{graphicx}
\usepackage{subcaption}

\begin{document}
	\section{Extended Tao-Eldrup model}
	We are given following equations from \cite{Zaleski2015}:
	\begin{equation}
		\label{eq1}
		P_{nm} = 
			\frac
				{\int\limits_{Z_{nm}\frac{R}{R+\Delta}}^{Z_{nm}}J_m(r)^2rdr}
				{\int\limits_0^{Z_{nm}}J_m(r)^2rdr}
	\end{equation}
	\begin{equation}
		\lambda_{nm} = \lambda_pP_{nm} + \lambda_i(1-P_{nm})
	\end{equation}
	\begin{equation}
		\tau =
			\frac
				{\sum\limits_n\sum\limits_mg_m\mathrm{exp}\left(-\frac{E_{nm}}{kT}\right)}
				{\sum\limits_n\sum\limits_mg_m\lambda_{nm}\mathrm{exp}\left(-\frac{E_{nm}}{kT}\right)}
	\end{equation}
	\begin{equation*}
		E_{nm} = \frac{\hbar^2}{4m_e}\frac{{Z_{nm}}^2}{(R+\Delta)^2}
	\end{equation*}
	where $Z_{nm}$ is the $n$-th node of the Bessel function of the
	first kind $J_m(r)$ and $g_m$ is the statistical weight of the $m$-th state
	($g_0 = 1$, $g_{m>0}=2$).
	\section{Bessel functions}
	\subsection{My idea of how to calculate things}
	\subsubsection{Approxmiating}
	Values of Bessel functions of the first kind are
	kindly provided by the Go's mathematics package.
	Because Bessel functions' derivatives can be defined by 
	\begin{equation*}
		\frac{d}{dx}J_n(x) = J_{n-1}(x) - \frac{n}{x}J_n(x)
	\end{equation*}
	or by
	\begin{equation*}
		\frac{d}{dx}J_n(x) = \frac{1}{2}(J_{n-1}(x) - J_{n+1}(x))
	\end{equation*}
	we can use approximation techniques for finding zeroes that use derivatives.
	I'm thinking mainly about Newton's or Halley's methodes.

	Recursive sequence in Newton's method is defined in our case by
	\begin{equation*}
		x_{n+1} = x_n - \frac{J_v(x_n)}{J'_v(x_n)} = 
			x_n - \frac{J_v(x_n)}{J_{v-1}(x) - \frac{v}{x_n}J_v(x)} =
			x_n - \frac{2J_v(x)}{J_{v-1}(x) - J_{v+1}(x)}
	\end{equation*}
	The last equation might be actually worse than I innitialy thought
	because we would need to calculate Bessel function 3 times compared to 2 times
	in the second to last equation.

	Halley's method can be defined by
	\begin{equation*}
		x_{n+1} = x_n - \frac{2J_v(x_n)J'_v(x_n)}{2(J'_v(x_n))^2 - J_v(x_n)J''_v(x_n)}
	\end{equation*}
	which looks not as easy by I will check how fast it converges compared to
	Newton's method.
	\subsubsection{Initial guess}
	This wasn't easy because you can't use finite equation for 
	it because it decreses precision with bigger values.

	For zeros of $J_0(x)$ I used approximation given by McMahon's
	asymptotic expansion \cite[equation 10.21.19]{NIST:DLMF}	
	\begin{equation}
		j_{v,m} \sim a - \frac{\mu - 1}{8a} - \frac{4(\mu -1)(7\mu-31)}{3(8a)^3}
			- \frac{32(\mu-1)(83\mu^2-982\mu+3779)}{15(8a)^5} - \dots
	\end{equation}
	where $j_{v,m}$ is the $m$-th zero of the Bessel function $J_v(x)$,
	$\mu=4v^2$ and $a=(m+\frac{1}{2}v-\frac{1}{4})\pi$.
	
	For zeros $j_{v,m}$ where $v > 0$ I just put it in bounds $j_{v-1,m-1} < j_{v,m} < j_{v-1,m+1}$
	\cite[equation 10.21.2]{NIST:DLMF} and approximated using Newton's formula until I got what I wanted. I check values for
	big $v$ and $m$ and got correct anserw so it is assumed that my code is working.
	\subsection{How it's done by GNU Scientific Library}
	\subsubsection{Value of Bessel function $J_n(x)$}
	\subsubsection{Zeros of Bessel function}
	\subsection{Numerical integration}
	Next mission is to calculate \ref{eq1} by using numerical 
	integration algorithms.
	Use of simple midpoint rule is inefficient so there's need to use Gauss'
	quadrature. I arbitrarily chose 10 points given in \cite[table 3.5.2]{NIST:DLMF}
	\section{A study of the extended Tao-Eldrup model's parameters}
	%% TODO: fill it with text
	\begin{figure}[!ht]
	\centering
	\scalebox{.9}{\input{params/temp/out.tex}}
	\caption{Temperature}
	\end{figure}
	\begin{figure}[!ht]
	\centering
	\scalebox{.9}{\input{params/res/out.tex}}
	\caption{Number of $n = m$ to sum}
	\end{figure}
	\begin{figure}[!ht]
	\centering
	\scalebox{.9}{\input{params/delta/out.tex}}
	\caption{$\Delta$ parameter}
	\end{figure}
	\section{Calculating $R$ from ETE}
	\subsection{Initial guess}
	I went for approxmiating $\tau$ as 
	\begin{equation}
		\tau \approx \frac{140}{1+x/5^{-3/2}},
	\end{equation}
	which is a CDF of a log-logistic distribution.

	It looks quite good but it doesn't matter so much as
	it will just get corrected.
	I will check later fo comparison against different approximation.
	I used wxMaxima to calculate inverse of this function which in my case
	will be 
	\begin{equation}
		R \approx \left(\frac{39480499x}{4993932(140-x)}\right)^{\textstyle\frac{2}{3}}
	\end{equation}
	\bibliographystyle{abbrvurl}
	\bibliography{refs.bib}
\end{document}